\documentclass[11pt,a4paper,spanish]{book}
%\usepackage{estilo_unir}
\usepackage{biblatex}
\usepackage[utf8]{inputenc}
\addbibresource{}

\begin{document}
	%---------------------------
	%título del trabajo y autor
	%---------------------------
	\title{Reconocimiento y clasificación de emociones en la lengua no aprendidas}
	\author{Luisa Sánchez Avivar}
	%\date{d de mes de 2019}
	%\director{CiroRodríguez}
	%\nombreciudad{Lausanne}
	
	%---------------------------
	%marges
	%---------------------------
	%\usepackage[margin=1.9cm]{geometry}
	%---------------------------
	%---------------------------
	%---------------------------
	%---------------------------
	
	\chapter{Objetivos y metodología de trabajo}
		
	\section{Objetivo General}
	Disponer de al menos 3 conjuntos de datos pertenecientes a 3 idiomas distintos con una estructura de etiquetado similar, de los cuales uno se tomará como referencia para realizar una clasificación emocional con un porcentaje de acierto de al menos  un 80%  
	
	\section{Objetivo específicos}
	Para conseguir el alcance establecido, es necesario que los siguientes puntos sean satisfechos:
	\begin{itemize}
		\item Hacer un estudio del estado del arte sobre diferentes métodos, técnicas, y conjunto de datos utilizados en el reconocimiento de emociones a través de la voz. Aquí también se explorará si se dispone de la documentación necesaria, cómo cada uno de esos métodos pudiera estar relacionado con la lengua que usa para aplicarlo y su fonética.
		
		\item  Conseguir al menos 3 datasets pertenecientes a 3 idiomas diferentes donde uno de ellos será usado como referencia, y además, deberán cumplir las siguientes condiciones: Uno de los conjuntos de datos restantes deberá tener raíces fonéticas distintas al corpus de referencia, y el otro tener raíces fonéticas similares.
		
		\item Diseñar una solución en la que el conjunto de datos de referencia tenga un porcentaje de acierto del 80\% en la clasificación de emociones.
		
		\item Aplicar el modelo diseñado en el paso anterior a los otros conjuntos de datos.
		
		\item Evaluar la tasa de acierto obtenida en cada uno de eso conjuntos y comparar los resultados obtenidos.
		
	\end{itemize}
	\section{Metodología de trabajo}
	
	\begin{comment}
	Metodologia:
		https://assaf-pinhasi.medium.com/towards-a-development-methodology-for-machine-learning-part-i-f1050a0bc607
	\end{comment}
	
	
	\printbibliography
	
\end{document}