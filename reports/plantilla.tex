\documentclass[11pt,a4paper,spanish]{book}
\usepackage{estilo_unir}



%---------------------------
%t�tulo del trabajo y autor
%---------------------------
\title{Escribir el t�tulo del documento}
\author{Nombre y Apellidos}
\date{d de mes de 2019}
\director{ nombre y apellidos}
\nombreciudad{Nombre ciudad}

%---------------------------
%marges
%---------------------------
%\usepackage[margin=1.9cm]{geometry}
%---------------------------
%---------------------------
%---------------------------
%---------------------------
\begin{document}
\renewcommand{\listfigurename}{�ndice de Ilustraciones}
\renewcommand{\listtablename}{�ndice de Tablas}
\renewcommand{\contentsname}{�ndice de Contenidos}
\renewcommand{\figurename}{Figura}
\renewcommand{\tablename}{Tabla} 

\maketitle

\frontmatter
\tableofcontents
\listoffigures
\listoftables

\chapter{Resumen}
{\bf Nota:} En este apartado se introducir� un breve resumen en espa�ol del trabajo realizado (extensi�n m�xima: 150 palabras). Este resumen debe incluir el objetivo o prop�sito de la investigaci�n, la metodolog�a, los resultados y las conclusiones.


{\bf Palabras Clave:} Se deben incluir de 3 a 5 palabras claves en espa�ol

\chapter{Abstract}
{\bf Nota:} En este apartado se introducir� un breve resumen en espa�ol del trabajo realizado (extensi�n m�xima: 150 palabras). Este resumen debe incluir el objetivo o prop�sito de la investigaci�n, la metodolog�a, los resultados y las conclusiones.


{\bf Palabras Clave:} Se deben incluir de 3 a 5 palabras claves en ingl�s




\mainmatter
\chapter{Introducci�n}

\chapter{Contexto y Estado del Arte}

\chapter{Identificaci�n de Requisitos}

\chapter{Objetivos}

\chapter{Desarrollo del trabajo}

\chapter{Conclusiones y Trabajo Futuro}

\begin{thebibliography}{a}
\bibitem{etiqueta} \textsc{Autores},
\textit{nombre referencia.}
Informaci�n addicional
\end{thebibliography}
%\bibliographystyle{plain} 
%\bibliography{bibliografia}

\appendix
\chapter{Apendices}
Atenci�n, deber� generar un pdf con la plantilla de art�culo y a�adirla como anexo utilizando includepdf.

%\includepdf[pages=-]{anexo.pdf} %descomentar cu�ndo est� el art�culo terminado.
\end{document}





















